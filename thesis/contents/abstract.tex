\begin{abstract}
\par 在当前复杂多变的市场环境中,公司面临着诸多挑战和机遇,分析公司经营情况,
并相应对公司是否会破产进行预测,对公司和投资者都至关重要。
本文通过分析台湾经济日报1999年至2009年间的数据,
运用多种统计分析和机器学习方法,对公司的经营状况进行分析,对破产与否进行预测。
\par 针对公司经营情况,\textit{KMO}检验和巴特利特球状检验结果表明\textbf{数据集适合
使用因子分析},在\textbf{因子分析模型中选择29个因子},利用\textbf{方差最大化}方法对载荷矩阵进行旋转,
以增强可解释性。利用\textbf{加权最小二乘法得到因子得分向量},并用\textbf{因子对应的方差贡献率进行
加权},得到公司得分进行排名,排名前三的公司编号依次为:5257,4023,2364。
\par 协方差矩阵表明破产标签与特征的线性相关性较弱,
使用\textbf{互信息衡量单个特征与标签之间的非线性相关性},并依据得分进行
特征重要性排名,\textbf{排名前三的特征依次为:借贷依赖、利息费用比率和资产负债率}。
但是选择的特征不都能从经济学中找到解释,反映出统计学角度选择特征的局限性。
\par 针对公司破产预测,选择使用\textbf{高斯判别分析、随机森林和集成学习}。互信息
选择的18个特征都\textbf{通过正态分布检验},符合\textit{GDA}的模型假设。最大化联合
似然函数得到参数估计,预测结果的混淆矩阵表明,在数据集不平衡的情况下,
\textit{GDA}能够找出20\%的破产公司,整体\textbf{准确率为40.7\%,F1值为0.28}。
\par 对预处理后的样本(保留57个方差$>0.15$,协方差$<0.8$的特征)进行\textbf{主成分分析,
选择27个主成分},累计贡献率达到90\%。利用主成分对原始样本进行降维,将产生
的数据划分为训练集和测试集,进一步\textbf{对训练集数据进行\textit{BorderlineSMOTE}
过采样},平衡后的数据集作为随机森林和集成学习模型的输入。
其中随机森林的\textbf{准确率为41.2\%,F1值为0.36},进一步观察随机森林对于特征的重要性
评分,\textbf{最重要的因素仍然是借贷依赖}。集成学习采用堆叠方法,用极度随机树、CatBoost、
XGBoost作为基分类器,随机森林作为元分类器,模型\textbf{准确率为
71.4\%,F1值0.20}。
\par 本文利用因子分析方法分析公司破产的影响因素,并在此基础上选择
特征,对公司是否破产进行预测,模型预测准确率高、可解释性好。
尽管没有考虑特征之间的高阶相关性,破产标签的预测F1值较低,但是模型
为分析公司运营情况、预测公司走向提供了良好的切入点。
\keywords{因子分析 \quad 随机森林 \quad 高斯判别分析 \quad 集成学习 \quad 公司经营情况}
\end{abstract}