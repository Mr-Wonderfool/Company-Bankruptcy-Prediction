\section{问题分析}
\subsection{数据集预处理}
\par 数据集除破产标记外共有95个属性,6819个元组。数据集具有一定的不平衡性,
在有监督分类时考虑采用\textit{SMOTE}过采样方法平衡数据集,
防止模型在训练过程中偏向于多数类样本。数据集还存在同一属性列取值均相同的异常数据,
考虑在特征筛选时将特征方差作为筛选标准,对于波动很小的特征予以去除。
\subsection{公司经营情况分析}
\par 对公司经营状况的评价涉及到评估标准的确定、算法的复杂性、评价结果的可解释性等问题,
确定一个可量化、可解释、简洁的合理评估标准是构建有效评估模型的首要挑战。
\par 对于特征和标签之前的非线性关系,考虑使用互信息计算特征得分,根据
得分对变量进行排序,考察排名前10的变量,使用经济学知识对变量
与公司经营情况的关系进行进一步分析。
\par 从降维的角度出发,使用因子分析方法,从研究原始变量相关矩阵内部的依赖关系出发,
试图以最少的信息丢失,把些具有复杂关系的95个变量归结为少量公共因子,在此基础上定量评价公司经营状况,
同时通过载荷矩阵旋转提高结果的可解释性。
\par 在分析破产因素的问题上,对于双变量相关性,考虑到数据分布和为了评估
各因素与破产之间的线性关系及关系紧密程度,计算
每个属性与Bankrupt属性的皮尔逊相关系数、肯德尔相关系数、斯皮尔曼相关系数,
特别关注在0.01级别相关性显著的属性。对于用除Bankrupt属性外的属性因子分析
得到的各个因子计算了每个因子与Bankrupt属性的肯德尔相关系数、
斯皮尔曼相关系数,以分析导致破产的因素。
\subsection{公司破产预测}
在结合历史数据,对将要破产公司进行预测的问题上,该预测问题是二分类问题,
首先需要考虑数据集的异常数据与平衡性问题,对异常数据进行剔除,
并使用SMOTE采样方法平衡数据集。特征处理可以采用互信息筛选和\textit{PCA}
两种方式,产生的特征输入到模型。
\par 二分类的预测模型有很多,考虑到特征众多、内部关系较为复杂,
选择逻辑回归、随机森林、集成学习模型对公司进行破产预测。利用决策树进行
分类后,利用分类过程中计算的信息增益,根据重要性对特征进行排序。